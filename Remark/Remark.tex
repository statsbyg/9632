\section{Remark}

Chemical, biological, or solar sensors can be networked together as a sensor network to strengthen the power of sensing. A sensor network is controlled through a software core engine. The network is typically wireless but may also be wired. Sensor networks are designed to be self-configuring such that they can gather information about a large geographical area or about movements of an object for surveillance purposes.

Sensor networks can be used for target tracking, environmental monitoring, system control, and chemical or biological detection. In military applications, sensor networks can enable soldiers to see around corners and to detect chemical and biological weapons long before they get close enough to cause harm. Civilian uses include environmental monitoring, traffic control, and providing health care monitoring for the elderly while allowing them more freedom to move about.

20.1.1. Clustering in Sensor Networks
The region being sensed is normally partitioned into equally loaded clusters of sensor nodes, as shown in Figure 20.1. A cluster in a sensor network resembles a domain in a computer network. In other words, nodes are inserted in the vicinity of a certain predefined region, forming a cluster. Different types of sensors can also be deployed in a region. Thus, a sensor network is typically cluster based and has irregular topology. The most effective routing scheme in sensor networks is normally based on the energy (battery level) of nodes. In such routing schemes, the best path has the highest amount of total energy. The network of such sensing nodes is constructed with identical sensor nodes, regardless of the size of the network. In Figure 20.1, three clusters are interconnected to the main base station, each cluster contains a cluster head responsible for routing data from its corresponding cluster to a base station.
\newpage
\begin{sidewaystable}[p]\small
\begin{tabular}{|l*{25}{|c}|r|}
\hline
Group&No&PID&\multicolumn{2}{c|}{Job}&\multicolumn{2}{c|}{RA}&\multicolumn{2}{c|}{SD}&\multicolumn{2}{c|}{ST}
&\multicolumn{2}{c|}{SI}&\multicolumn{2}{c|}{SAM}&\multicolumn{2}{c|}{DB}&\multicolumn{2}{c|}{DB}
&\multicolumn{2}{c|}{UC}&\multicolumn{2}{c|}{AT}&\multicolumn{2}{c|}{MUC}\\
\cline{2-25}
&&&Value&Rank&Value&Rank&Value&Rank&Value&Rank&Value&Rank&Value&Rank&Value&Rank&Value&Rank&Value&Rank&Value&Rank&Value&Rank\\
\cline{1-23}
\hline
\multirow{5}{*}{XA-YB}&1&604&3&&&&&&&&&&&&&&&&&&&&&\\
\cline{2-25}&&&&&&&&&&&&&&&&&&&&&&&&\\
\cline{2-25}&&&&&&&&&&&&&&&&&&&&&&&&\\
\cline{2-25}&&&&&&&&&&&&&&&&&&&&&&&&\\
\cline{2-25}&&&&&&&&&&&&&&&&&&&&&&&&\\
\hline
\multicolumn{2}{|c}{Average}&&&&&&&&&&&&&&&&&&&&&&&\\\hline
\multirow{5}{*}{XA-YB}&1&604&3&&&&&&&&&&&&&&&&&&&&&\\
\cline{2-25}&&&&&&&&&&&&&&&&&&&&&&&&\\
\cline{2-25}&&&&&&&&&&&&&&&&&&&&&&&&\\
\cline{2-25}&&&&&&&&&&&&&&&&&&&&&&&&\\
\cline{2-25}&&&&&&&&&&&&&&&&&&&&&&&&\\
\hline
\multicolumn{2}{|c}{Average}&&&&&&&&&&&&&&&&&&&&&&&\\\hline
\multirow{5}{*}{XA-YB}&1&604&3&&&&&&&&&&&&&&&&&&&&&\\
\cline{2-25}&&&&&&&&&&&&&&&&&&&&&&&&\\
\cline{2-25}&&&&&&&&&&&&&&&&&&&&&&&&\\
\cline{2-25}&&&&&&&&&&&&&&&&&&&&&&&&\\
\cline{2-25}&&&&&&&&&&&&&&&&&&&&&&&&\\
\hline
\multicolumn{2}{|c}{Average}&&&&&&&&&&&&&&&&&&&&&&&\\\hline
\multirow{6}{*}{XA-YB}&1&604&3&&&&&&&&&&&&&&&&&&&&&\\
\cline{2-25}&&&&&&&&&&&&&&&&&&&&&&&&\\
\cline{2-25}&&&&&&&&&&&&&&&&&&&&&&&&\\
\cline{2-25}&&&&&&&&&&&&&&&&&&&&&&&&\\
\cline{2-25}&&&&&&&&&&&&&&&&&&&&&&&&\\
\cline{2-25}&&&&&&&&&&&&&&&&&&&&&&&&\\
\hline
\multicolumn{2}{|l}{Average}&&&&&&&&&&&&&&&&&&&&&&&\\\hline
\multicolumn{2}{|c}{T.Ave}&&&&&&&&&&&&&&&&&&&&&&&\\\hline
\multicolumn{2}{|l}{Standard Deviation}&&\\
\cline{1-4}
\multicolumn{2}{|l}{Minimum}&&\\
\cline{1-4}
\multicolumn{2}{|l}{Maximum}&&\\
\hline
\end{tabular}
\end{sidewaystable}





