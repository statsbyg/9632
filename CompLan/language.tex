\section{language}
Computer and communication networks to provide a wide range of services, from simple networks of computers to remote-file access, digital libraries, videoconferencing, and networking billions of users and devices. Before exploring the world of computer and communication networks, we need to study the fundamentals of packet-switched networks as the first step. Packet-switched networks are the backbone of the data communication infrastructure. Therefore, our focus in this chapter is on the big picture and the conceptual aspects of this backbone:

Basic definitions in data networks

Types of packet-switched networks

Packet size and optimizations

We start with the basic definitions and fundamental concepts, such as messages, packets, and frames, and packet switching versus circuit switching. We learn what the Internet is and how Internet service providers (ISPs) are formed. We then proceed to types of packet-switched networks and how a message can be handled by either connection-oriented networks or connectionless networks. Because readers must get a good understanding of packets as data units, packet size and optimizations are also discussed.

\subsection{Scripting language}
Communication networks have become essential media for homes and businesses. The design of modern computer and communication networks must meet all the requirements for new communications applications. A ubiquitous broadband network is the goal of the networking industry. Communication services need to be available anywhere and anytime. The broadband network is required to support the exchange of multiple types of information, such as voice, video, and data, among multiple types of users, while satisfying the performance requirement of each individual application. Consequently, the expanding diversity of high-bandwidth communication applications calls for a unified, flexible, and efficient network. The design goal of modern communication networks is to meet all the networking demands and to integrate capabilities of networks in a broadband network.


\subsubsection{JavaScript}
Packet-switched networks are the building blocks of computer communication systems in which data units known as packets flow across networks. The goal of a broadband packet-switched network is to provide flexible communication in handling all kinds of connections for a wide range of applications, such as telephone calls, data transfer, teleconferencing, video broadcasting, and distributed data processing. One obvious example for the form of traffic is multirate connections, whereby traffic containing several different bit rates flows to a communication node. The form of information in packet-switched networks is always digital bits. This kind of communication infrastructure is a significant improvement over the traditional telephone networks known as circuit-switched networks.
\subsubsection{PHP}
Circuit-switched networks, as the basis of conventional telephone systems, were the only existing personal communication infrastructures prior to the invention of packet-switched networks. In the new communication structure, voice and computer data are treated the same, and both are handled in a unified network known as a packet-switched network, or simply an integrated data network. In conventional telephone networks, a circuit between two users must be established for a communication to occur. Circuit-switched networks require resources to be reserved for each pair of end users. This implies that no other users can use the already dedicated resources for the duration of network use. The reservation of network resources for each user results in an inefficient use of bandwidth for applications in which information transfer is bursty.


\subsubsection{VBScript}
Packet-switched networks with a unified, integrated data network infrastructure known as the Internet can provide a variety of communication services requiring different bandwidths. The advantage of having a unified, integrated data network is the flexibility to handle existing and future services with remarkably better performance and higher economical resource utilizations. An integrated data network can also derive the benefits of central network management, operation, and maintenance. Numerous requirements for integrated packed-switched networks are explored in later chapters:

Having robust routing protocols capable of adapting to dynamic changes in network topology

Maximizing the utilization of network resources for the integration of all types of services

Providing quality of service to users by means of priority and scheduling

Enforcing effective congestion-control mechanisms that can minimize dropping packets


\subsection{Programming Language}
Circuit-switched networking is preferred for real-time applications. However, the use of packet-switched networks, especially for the integration and transmission of voice and data, results in the far more efficient utilization of available bandwidth. Network resources can be shared among other eligible users. Packet-switched networks can span a large geographical area and comprise a web of switching nodes interconnected through transmission links. A network provides links among multiple users facilitating the transfer of information. To make efficient use of available resources, packet-switched networks dynamically allocate resources only when required.


\subsubsection{Interpreted}
A packet-switched network is organized as a multilevel hierarchy. In such networks, digital messages are fragmented into one or more smaller units of messages, each appended with a header to specify control information, such as the source and the destination addresses. This new unit of formatted message is called a packet, as shown in Figure 1.1. Packets are forwarded to a data network to be delivered to their destinations. In some circumstances, packets are also required to be attached together or further fragmented, forming a new packet known as a frame. Sometimes, a frame may be required to have multiple headers to carry out multiple tasks in multiple layers of a network, as shown in the figure.


\subsubsection{Compiled}
The Internet is the collection of hardware and software components that make up our global communication network. The Internet is indeed a collaboration of interconnected communication vehicles that can network all connected communicating devices and equipment and provide services to all distributed applications. It is almost impossible to plot an exact representation of the Internet, since it is continuously being expanded or altered. One way of imagining the Internet is shown in Figure 1.3, which illustrates a big-picture view of the worldwide computer network.


\subsection{System Programming}
To connect to the Internet, users need the services of an Internet service provider. Each country has international or national service providers, regional service providers, and local service providers. At the top of the hierarchy, national Internet service providers connect nations or provinces together. The traffic between each two national ISPs is very heavy. Two such ISPs are connected together through complex switching stations called network access points (NAPs). Each NAP has its own system administrator.

In contrast, regional Internet service providers are smaller ISPs connected to a national ISP in a hierarchical chart. A router can operate as a device to connect to ISPs. Routers operate on the basis of one or more common routing protocols. In computer networks, the entities must agree on a protocol, a set of rules governing data communications and defining when and how two users can communicate with each other.
\newline
\begin{table}[htb]
\caption{Ideentify teamviewe}
\begin{center}
\begin{tabular}{|r|r|}


\hline
$System program$&$Language$\\
\hline
Java&Java Script\\
VB&Visual Programming\\
HTML&Web page\\
C++&C code\\ \hline

\end{tabular}
\end{center}
\label{table:yared}
\end{table}





